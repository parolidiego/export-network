% Options for packages loaded elsewhere
\PassOptionsToPackage{unicode}{hyperref}
\PassOptionsToPackage{hyphens}{url}
%
\documentclass[
]{article}
\usepackage{amsmath,amssymb}
\usepackage{iftex}
\ifPDFTeX
  \usepackage[T1]{fontenc}
  \usepackage[utf8]{inputenc}
  \usepackage{textcomp} % provide euro and other symbols
\else % if luatex or xetex
  \usepackage{unicode-math} % this also loads fontspec
  \defaultfontfeatures{Scale=MatchLowercase}
  \defaultfontfeatures[\rmfamily]{Ligatures=TeX,Scale=1}
\fi
\usepackage{lmodern}
\ifPDFTeX\else
  % xetex/luatex font selection
\fi
% Use upquote if available, for straight quotes in verbatim environments
\IfFileExists{upquote.sty}{\usepackage{upquote}}{}
\IfFileExists{microtype.sty}{% use microtype if available
  \usepackage[]{microtype}
  \UseMicrotypeSet[protrusion]{basicmath} % disable protrusion for tt fonts
}{}
\makeatletter
\@ifundefined{KOMAClassName}{% if non-KOMA class
  \IfFileExists{parskip.sty}{%
    \usepackage{parskip}
  }{% else
    \setlength{\parindent}{0pt}
    \setlength{\parskip}{6pt plus 2pt minus 1pt}}
}{% if KOMA class
  \KOMAoptions{parskip=half}}
\makeatother
\usepackage{xcolor}
\usepackage[margin=1in]{geometry}
\usepackage{color}
\usepackage{fancyvrb}
\newcommand{\VerbBar}{|}
\newcommand{\VERB}{\Verb[commandchars=\\\{\}]}
\DefineVerbatimEnvironment{Highlighting}{Verbatim}{commandchars=\\\{\}}
% Add ',fontsize=\small' for more characters per line
\usepackage{framed}
\definecolor{shadecolor}{RGB}{248,248,248}
\newenvironment{Shaded}{\begin{snugshade}}{\end{snugshade}}
\newcommand{\AlertTok}[1]{\textcolor[rgb]{0.94,0.16,0.16}{#1}}
\newcommand{\AnnotationTok}[1]{\textcolor[rgb]{0.56,0.35,0.01}{\textbf{\textit{#1}}}}
\newcommand{\AttributeTok}[1]{\textcolor[rgb]{0.13,0.29,0.53}{#1}}
\newcommand{\BaseNTok}[1]{\textcolor[rgb]{0.00,0.00,0.81}{#1}}
\newcommand{\BuiltInTok}[1]{#1}
\newcommand{\CharTok}[1]{\textcolor[rgb]{0.31,0.60,0.02}{#1}}
\newcommand{\CommentTok}[1]{\textcolor[rgb]{0.56,0.35,0.01}{\textit{#1}}}
\newcommand{\CommentVarTok}[1]{\textcolor[rgb]{0.56,0.35,0.01}{\textbf{\textit{#1}}}}
\newcommand{\ConstantTok}[1]{\textcolor[rgb]{0.56,0.35,0.01}{#1}}
\newcommand{\ControlFlowTok}[1]{\textcolor[rgb]{0.13,0.29,0.53}{\textbf{#1}}}
\newcommand{\DataTypeTok}[1]{\textcolor[rgb]{0.13,0.29,0.53}{#1}}
\newcommand{\DecValTok}[1]{\textcolor[rgb]{0.00,0.00,0.81}{#1}}
\newcommand{\DocumentationTok}[1]{\textcolor[rgb]{0.56,0.35,0.01}{\textbf{\textit{#1}}}}
\newcommand{\ErrorTok}[1]{\textcolor[rgb]{0.64,0.00,0.00}{\textbf{#1}}}
\newcommand{\ExtensionTok}[1]{#1}
\newcommand{\FloatTok}[1]{\textcolor[rgb]{0.00,0.00,0.81}{#1}}
\newcommand{\FunctionTok}[1]{\textcolor[rgb]{0.13,0.29,0.53}{\textbf{#1}}}
\newcommand{\ImportTok}[1]{#1}
\newcommand{\InformationTok}[1]{\textcolor[rgb]{0.56,0.35,0.01}{\textbf{\textit{#1}}}}
\newcommand{\KeywordTok}[1]{\textcolor[rgb]{0.13,0.29,0.53}{\textbf{#1}}}
\newcommand{\NormalTok}[1]{#1}
\newcommand{\OperatorTok}[1]{\textcolor[rgb]{0.81,0.36,0.00}{\textbf{#1}}}
\newcommand{\OtherTok}[1]{\textcolor[rgb]{0.56,0.35,0.01}{#1}}
\newcommand{\PreprocessorTok}[1]{\textcolor[rgb]{0.56,0.35,0.01}{\textit{#1}}}
\newcommand{\RegionMarkerTok}[1]{#1}
\newcommand{\SpecialCharTok}[1]{\textcolor[rgb]{0.81,0.36,0.00}{\textbf{#1}}}
\newcommand{\SpecialStringTok}[1]{\textcolor[rgb]{0.31,0.60,0.02}{#1}}
\newcommand{\StringTok}[1]{\textcolor[rgb]{0.31,0.60,0.02}{#1}}
\newcommand{\VariableTok}[1]{\textcolor[rgb]{0.00,0.00,0.00}{#1}}
\newcommand{\VerbatimStringTok}[1]{\textcolor[rgb]{0.31,0.60,0.02}{#1}}
\newcommand{\WarningTok}[1]{\textcolor[rgb]{0.56,0.35,0.01}{\textbf{\textit{#1}}}}
\usepackage{graphicx}
\makeatletter
\def\maxwidth{\ifdim\Gin@nat@width>\linewidth\linewidth\else\Gin@nat@width\fi}
\def\maxheight{\ifdim\Gin@nat@height>\textheight\textheight\else\Gin@nat@height\fi}
\makeatother
% Scale images if necessary, so that they will not overflow the page
% margins by default, and it is still possible to overwrite the defaults
% using explicit options in \includegraphics[width, height, ...]{}
\setkeys{Gin}{width=\maxwidth,height=\maxheight,keepaspectratio}
% Set default figure placement to htbp
\makeatletter
\def\fps@figure{htbp}
\makeatother
\setlength{\emergencystretch}{3em} % prevent overfull lines
\providecommand{\tightlist}{%
  \setlength{\itemsep}{0pt}\setlength{\parskip}{0pt}}
\setcounter{secnumdepth}{-\maxdimen} % remove section numbering
\ifLuaTeX
  \usepackage{selnolig}  % disable illegal ligatures
\fi
\usepackage{bookmark}
\IfFileExists{xurl.sty}{\usepackage{xurl}}{} % add URL line breaks if available
\urlstyle{same}
\hypersetup{
  pdftitle={Social network analysis hw1: export-network},
  pdfauthor={Yijia Lin, Diego Paroli},
  hidelinks,
  pdfcreator={LaTeX via pandoc}}

\title{Social network analysis hw1: export-network}
\author{Yijia Lin, Diego Paroli}
\date{2025-04-23}

\begin{document}
\maketitle

\section{Libraries}\label{libraries}

\begin{Shaded}
\begin{Highlighting}[]
\FunctionTok{rm}\NormalTok{(}\AttributeTok{list =} \FunctionTok{ls}\NormalTok{())}
\FunctionTok{library}\NormalTok{(tidyverse)}
\FunctionTok{library}\NormalTok{(httr2)}
\FunctionTok{library}\NormalTok{(igraph)}
\FunctionTok{library}\NormalTok{(tidygraph)}
\FunctionTok{library}\NormalTok{(ggraph)}
\FunctionTok{library}\NormalTok{(visNetwork)}
\end{Highlighting}
\end{Shaded}

\section{Get the data}\label{get-the-data}

I have commented below to avoid calling it every time we render the
document

\begin{Shaded}
\begin{Highlighting}[]
\CommentTok{\# zip\_data \textless{}{-} request("https://networks.skewed.de/net/product\_space/files/SITC.csv.zip") |\textgreater{}}
\CommentTok{\#   req\_perform()}
\CommentTok{\# }
\CommentTok{\# writeBin(resp\_body\_raw(zip\_data), "exports\_SITC.csv.zip")}
\CommentTok{\# }
\CommentTok{\# unzip("exports\_SITC.csv.zip", exdir = "network{-}data")}
\CommentTok{\# }
\CommentTok{\# file.remove("exports\_SITC.csv.zip", )}
\end{Highlighting}
\end{Shaded}

\begin{Shaded}
\begin{Highlighting}[]
\NormalTok{nodes }\OtherTok{\textless{}{-}} \FunctionTok{read\_csv}\NormalTok{(}\StringTok{"network{-}data/nodes.csv"}\NormalTok{)}
\NormalTok{links }\OtherTok{\textless{}{-}} \FunctionTok{read\_csv}\NormalTok{(}\StringTok{"network{-}data/edges.csv"}\NormalTok{)}
\end{Highlighting}
\end{Shaded}

\begin{Shaded}
\begin{Highlighting}[]
\FunctionTok{head}\NormalTok{(nodes)}
\end{Highlighting}
\end{Shaded}

\begin{verbatim}
## # A tibble: 6 x 9
##   `# index`   pid community  size pos                  leamer name  color `_pos`
##       <dbl> <dbl>     <dbl> <dbl> <chr>                 <dbl> <chr> <chr> <chr> 
## 1         0  6932         0  48.8 array([4551.8996582~      8 WIRE~ "#9c~ array~
## 2         1  7362         0  65.2 array([ 216.8350982~      9 META~ "#40~ array~
## 3         2  7911         0  54.0 array([ 538.9149017~      9 RAIL~ "#40~ array~
## 4         3  8946         0  57.7 array([ 696.3942565~      7 NON-~ "#40~ array~
## 5         4  7264         0  73.3 array([  57.2840652~      9 PRIN~ "#40~ array~
## 6         5  2783         0  58.3 array([4662.2502441~      2 COMM~ "#ff~ array~
\end{verbatim}

\begin{Shaded}
\begin{Highlighting}[]
\FunctionTok{head}\NormalTok{(links)}
\end{Highlighting}
\end{Shaded}

\begin{verbatim}
## # A tibble: 6 x 4
##   `# source` target width color      
##        <dbl>  <dbl> <dbl> <chr>      
## 1          1    328  5.58 "#727272\n"
## 2          4    475  6.36 "#7b7b7b\n"
## 3          6     69  5.71 "#737373\n"
## 4          8     18  5.12 "#6c6c6c\n"
## 5          8      9  3.72 "#545454\n"
## 6         10    480  8.92 "#949494\n"
\end{verbatim}

\section{Description of the dataset}\label{description-of-the-dataset}

Network of economic products, where a pair of products are connected if
they are exported at similar rates by the same countries. The data are a
projection from a bipartite network of nations and the products they
export. Edges weights represent a similarity score (called
``proximity''). Data based on UN Comtrade worldwide trade patterns. SITC
network based on the Standard International Trade Classification.

\subsubsection{Properties:}\label{properties}

Weighted, Undirected

\subsubsection{Graph:}\label{graph}

\begin{Shaded}
\begin{Highlighting}[]
\NormalTok{graph }\OtherTok{\textless{}{-}} \FunctionTok{graph\_from\_data\_frame}\NormalTok{(links, }\AttributeTok{directed =} \ConstantTok{FALSE}\NormalTok{, }\AttributeTok{vertices =}\NormalTok{ nodes)}
\NormalTok{graph}
\end{Highlighting}
\end{Shaded}

\begin{verbatim}
## IGRAPH 15427f0 UN-- 774 1779 -- 
## + attr: name (v/c), pid (v/n), community (v/n), size (v/n), pos (v/c),
## | leamer (v/n), color (v/c), _pos (v/c), width (e/n), color (e/c)
## + edges from 15427f0 (vertex names):
## [1] METAL FORMING MACHINE TOOLS                    --CONVERTERS,LADLES,INGOT MOULDS AND CASTING MACH.  
## [2] PRINTING PRESSES                               --OTHER MACH.-TOOLS FOR WORKING METAL OR MET.CARBIDE
## [3] OTHER FOOD PROCESSING MACHINERY AND PARTS      --PARTS OF THE MACHINERY OF 744.2-                  
## [4] PRODUCER GAS AND WATER GAS GENERATORS AND PARTS--OTHER PUMPS FOR LIQUIDS & LIQUID ELEVATORS        
## [5] PRODUCER GAS AND WATER GAS GENERATORS AND PARTS--CINEMATOGRAPHIC CAMERAS,PROJECTORS,SOUND-REC,PAR  
## + ... omitted several edges
\end{verbatim}

\section{Questions}\label{questions}

\subsection{1. What is the number of nodes and
links?}\label{what-is-the-number-of-nodes-and-links}

\begin{Shaded}
\begin{Highlighting}[]
\FunctionTok{vcount}\NormalTok{(graph)}
\end{Highlighting}
\end{Shaded}

\begin{verbatim}
## [1] 774
\end{verbatim}

\begin{Shaded}
\begin{Highlighting}[]
\FunctionTok{ecount}\NormalTok{(graph)}
\end{Highlighting}
\end{Shaded}

\begin{verbatim}
## [1] 1779
\end{verbatim}

There are in total 774 nodes and 1779 links in this network.

\subsection{2. What is the average degree in the network? And the
standard deviation of the
degree?}\label{what-is-the-average-degree-in-the-network-and-the-standard-deviation-of-the-degree}

\begin{Shaded}
\begin{Highlighting}[]
\FunctionTok{mean}\NormalTok{(}\FunctionTok{degree}\NormalTok{(graph))}
\end{Highlighting}
\end{Shaded}

\begin{verbatim}
## [1] 4.596899
\end{verbatim}

\begin{Shaded}
\begin{Highlighting}[]
\FunctionTok{sd}\NormalTok{(}\FunctionTok{degree}\NormalTok{(graph))}
\end{Highlighting}
\end{Shaded}

\begin{verbatim}
## [1] 5.994848
\end{verbatim}

The average degree is 4.5969 in this network, with a standard deviation
of 5.9948.

\subsection{3. Plot the degree distribution in linear-linear scale and
in log-log-scale. Does it have a typical connectivity? What is the
degree of the most connected
node?}\label{plot-the-degree-distribution-in-linear-linear-scale-and-in-log-log-scale.-does-it-have-a-typical-connectivity-what-is-the-degree-of-the-most-connected-node}

\begin{Shaded}
\begin{Highlighting}[]
\CommentTok{\# In linear{-}linear scale}
\FunctionTok{ggplot}\NormalTok{() }\SpecialCharTok{+} 
  \FunctionTok{geom\_histogram}\NormalTok{(}\FunctionTok{aes}\NormalTok{(}\AttributeTok{x =} \FunctionTok{degree}\NormalTok{(graph)),}
                 \AttributeTok{fill =} \StringTok{"\#69b3a2"}\NormalTok{, }\AttributeTok{color =} \StringTok{"white"}\NormalTok{, }\AttributeTok{alpha =} \FloatTok{0.8}\NormalTok{) }\SpecialCharTok{+} 
  \FunctionTok{labs}\NormalTok{(}\AttributeTok{x =} \StringTok{"Degree"}\NormalTok{, }\AttributeTok{y =} \StringTok{""}\NormalTok{, }\AttributeTok{title =} \StringTok{"Degree distribution in linear{-}linear scale"}\NormalTok{)}\SpecialCharTok{+}
  \FunctionTok{theme\_minimal}\NormalTok{(}\AttributeTok{base\_size =} \DecValTok{14}\NormalTok{) }\SpecialCharTok{+}
  \FunctionTok{theme}\NormalTok{(}
    \AttributeTok{plot.title =} \FunctionTok{element\_text}\NormalTok{(}\AttributeTok{face =} \StringTok{"bold"}\NormalTok{, }\AttributeTok{hjust =} \FloatTok{0.5}\NormalTok{),}
    \AttributeTok{axis.title =} \FunctionTok{element\_text}\NormalTok{(}\AttributeTok{face =} \StringTok{"bold"}\NormalTok{)}
\NormalTok{  )}
\end{Highlighting}
\end{Shaded}

\begin{verbatim}
## `stat_bin()` using `bins = 30`. Pick better value with `binwidth`.
\end{verbatim}

\includegraphics{export-network_files/figure-latex/unnamed-chunk-8-1.pdf}

\begin{Shaded}
\begin{Highlighting}[]
\CommentTok{\# In log{-}log scale}
\NormalTok{deg }\OtherTok{\textless{}{-}} \FunctionTok{degree}\NormalTok{(graph)}
\FunctionTok{ggplot}\NormalTok{(}\FunctionTok{data.frame}\NormalTok{(}\AttributeTok{deg =}\NormalTok{ deg), }\FunctionTok{aes}\NormalTok{(}\AttributeTok{x =}\NormalTok{ deg)) }\SpecialCharTok{+}
  \FunctionTok{geom\_histogram}\NormalTok{(}\AttributeTok{fill =} \StringTok{"\#69b3a2"}\NormalTok{, }\AttributeTok{color =} \StringTok{"white"}\NormalTok{, }\AttributeTok{alpha =} \FloatTok{0.8}\NormalTok{) }\SpecialCharTok{+}
  \FunctionTok{scale\_x\_log10}\NormalTok{() }\SpecialCharTok{+}
  \FunctionTok{scale\_y\_log10}\NormalTok{() }\SpecialCharTok{+}
  \FunctionTok{labs}\NormalTok{(}
    \AttributeTok{x =} \StringTok{"Degree (log scale)"}\NormalTok{,}
    \AttributeTok{y =} \StringTok{"Frequency (log scale)"}\NormalTok{,}
    \AttributeTok{title =} \StringTok{"Degree Distribution in Log{-}Log Scale"}
\NormalTok{  ) }\SpecialCharTok{+}
  \FunctionTok{theme\_minimal}\NormalTok{(}\AttributeTok{base\_size =} \DecValTok{14}\NormalTok{) }\SpecialCharTok{+}
  \FunctionTok{theme}\NormalTok{(}
    \AttributeTok{plot.title =} \FunctionTok{element\_text}\NormalTok{(}\AttributeTok{face =} \StringTok{"bold"}\NormalTok{, }\AttributeTok{hjust =} \FloatTok{0.5}\NormalTok{),}
    \AttributeTok{axis.title =} \FunctionTok{element\_text}\NormalTok{(}\AttributeTok{face =} \StringTok{"bold"}\NormalTok{)}
\NormalTok{  )}
\end{Highlighting}
\end{Shaded}

\begin{verbatim}
## `stat_bin()` using `bins = 30`. Pick better value with `binwidth`.
\end{verbatim}

\begin{verbatim}
## Warning in scale_y_log10(): log-10 transformation introduced infinite values.
\end{verbatim}

\begin{verbatim}
## Warning: Removed 11 rows containing missing values or values outside the scale range
## (`geom_bar()`).
\end{verbatim}

\includegraphics{export-network_files/figure-latex/unnamed-chunk-8-2.pdf}

\begin{Shaded}
\begin{Highlighting}[]
\CommentTok{\# Max\_degree}
\FunctionTok{max\_degree}\NormalTok{(graph)}
\end{Highlighting}
\end{Shaded}

\begin{verbatim}
## [1] 43
\end{verbatim}

We can observe that this network \textbf{does not exhibit typical
connectivity}: Its degree distribution is highly skewed and lacks a
clear peak. Most nodes have a very low degree, while a few have very
high degree. In the log-log scale plot, we can observe a power-law-like
distribution, which is not consistent with a Poisson-like distribution,
where most nodes would have approximately the same number of links and
no hubs.

The most connected node here has a degree of 43.

\subsection{4. What is the clustering coefficient (transitivity) in the
network?}\label{what-is-the-clustering-coefficient-transitivity-in-the-network}

\begin{Shaded}
\begin{Highlighting}[]
\FunctionTok{transitivity}\NormalTok{(graph)}
\end{Highlighting}
\end{Shaded}

\begin{verbatim}
## [1] 0.429691
\end{verbatim}

The global transitivity of this network is 0.4297, which is closer to 0
than to 1, indicating a relatively low tendency of clustering.

\subsection{5. What is the assortativity (degree) in the
network?}\label{what-is-the-assortativity-degree-in-the-network}

\begin{Shaded}
\begin{Highlighting}[]
\FunctionTok{assortativity\_degree}\NormalTok{(graph)}
\end{Highlighting}
\end{Shaded}

\begin{verbatim}
## [1] 0.4571059
\end{verbatim}

The assortativity coefficient of this network is 0.4571 (greater than
0), indicating a moderate to strong tendency for nodes to connect with
others that have a similar degree. In other words, high-degree nodes
tend to connect with other high-degree nodes, and low-degree nodes tend
to connect with other low-degree nodes. This is a sign of assortative
mixing.

\subsection{6. Using the Louvain method, does the network have a
community structure? If so, what is its
modularity?}\label{using-the-louvain-method-does-the-network-have-a-community-structure-if-so-what-is-its-modularity}

\begin{Shaded}
\begin{Highlighting}[]
\NormalTok{louvain\_cluster }\OtherTok{\textless{}{-}} \FunctionTok{cluster\_louvain}\NormalTok{(graph, }\AttributeTok{weights =} \FunctionTok{E}\NormalTok{(graph)}\SpecialCharTok{$}\NormalTok{width)}

\FunctionTok{sizes}\NormalTok{(louvain\_cluster)}
\end{Highlighting}
\end{Shaded}

\begin{verbatim}
## Community sizes
##   1   2   3   4   5   6   7   8   9  10  11  12  13  14  15  16  17  18  19  20 
##  57 110  10  58 140  46  66  97  14   8  18   4  17  16   3  13  13   8  16  11 
##  21  22  23  24  25  26  27  28  29 
##   6   8   6   9   6   3   5   3   3
\end{verbatim}

\begin{Shaded}
\begin{Highlighting}[]
\FunctionTok{modularity}\NormalTok{(louvain\_cluster)}
\end{Highlighting}
\end{Shaded}

\begin{verbatim}
## [1] 0.7407729
\end{verbatim}

\begin{Shaded}
\begin{Highlighting}[]
\CommentTok{\# Only display labels for nodes of a degree higher than or equal to 30}
\FunctionTok{V}\NormalTok{(graph)}\SpecialCharTok{$}\NormalTok{label }\OtherTok{\textless{}{-}} \FunctionTok{ifelse}\NormalTok{(deg }\SpecialCharTok{\textgreater{}=} \DecValTok{30}\NormalTok{, }\FunctionTok{V}\NormalTok{(graph)}\SpecialCharTok{$}\NormalTok{name, }\ConstantTok{NA}\NormalTok{)}
\CommentTok{\# Generate random angles for lables to avoid overlapping}
\NormalTok{random\_angles }\OtherTok{\textless{}{-}} \FunctionTok{runif}\NormalTok{(}\FunctionTok{length}\NormalTok{(}\FunctionTok{V}\NormalTok{(graph)), }\DecValTok{0}\NormalTok{, }\DecValTok{2} \SpecialCharTok{*}\NormalTok{ pi)}
\CommentTok{\# Plotting}
\FunctionTok{plot}\NormalTok{(louvain\_cluster, graph,}
     \AttributeTok{vertex.label =} \FunctionTok{V}\NormalTok{(graph)}\SpecialCharTok{$}\NormalTok{label,}
     \AttributeTok{vertex.label.cex =} \FloatTok{0.7}\NormalTok{,}
     \AttributeTok{vertex.label.color =} \StringTok{"black"}\NormalTok{,}
     \AttributeTok{vertex.label.dist =} \DecValTok{2}\NormalTok{,  }
     \AttributeTok{vertex.label.degree =}\NormalTok{ random\_angles  )}
\end{Highlighting}
\end{Shaded}

\includegraphics{export-network_files/figure-latex/unnamed-chunk-11-1.pdf}

Yes, the network has a clear community structure. Using the Louvain
method, the network was partitioned into multiple communities, as the
plot indicates. The modularity value is \textbf{0.7466}, which is
considered high and indicates a strong modular (community) structure
within the network.

\subsection{7. Test that the clustering coefficient in the network
cannot be statistically explain by a configuration model in which the
nodes have the same degree distribution as the
original.}\label{test-that-the-clustering-coefficient-in-the-network-cannot-be-statistically-explain-by-a-configuration-model-in-which-the-nodes-have-the-same-degree-distribution-as-the-original.}

\begin{Shaded}
\begin{Highlighting}[]
\NormalTok{original\_clustering }\OtherTok{\textless{}{-}} \FunctionTok{transitivity}\NormalTok{(graph, }\AttributeTok{type =} \StringTok{"global"}\NormalTok{)}
\end{Highlighting}
\end{Shaded}

\begin{Shaded}
\begin{Highlighting}[]
\CommentTok{\# Create 100 random configuration models and register the clustering coefficient}
\NormalTok{num\_simulations }\OtherTok{\textless{}{-}} \DecValTok{100}
\NormalTok{simulated\_clustering }\OtherTok{\textless{}{-}} \FunctionTok{numeric}\NormalTok{(num\_simulations)}
\ControlFlowTok{for}\NormalTok{ (i }\ControlFlowTok{in} \DecValTok{1}\SpecialCharTok{:}\NormalTok{num\_simulations) \{}
\NormalTok{  config\_graph }\OtherTok{\textless{}{-}} \FunctionTok{sample\_degseq}\NormalTok{(}\FunctionTok{degree}\NormalTok{(graph), }\AttributeTok{method =} \StringTok{"vl"}\NormalTok{)  }
\NormalTok{  simulated\_clustering[i] }\OtherTok{\textless{}{-}} \FunctionTok{transitivity}\NormalTok{(config\_graph, }\AttributeTok{type =} \StringTok{"global"}\NormalTok{)}
\NormalTok{\}}
\end{Highlighting}
\end{Shaded}

Here we use the method ``vl'' (Viger--Latapy) instead of the traditional
one, ``configuration'', in order to avoid multi-edges and self-loops. As
a consequence, we don't need to add too many more checks in the loop
like is.simple().

\begin{Shaded}
\begin{Highlighting}[]
\CommentTok{\# Evaluation and statistical test}
\CommentTok{\# Calculate the p value:clustering coefficient of the simulated network VS random configuration model}
\NormalTok{p\_value }\OtherTok{\textless{}{-}} \FunctionTok{mean}\NormalTok{(simulated\_clustering }\SpecialCharTok{\textgreater{}=}\NormalTok{ original\_clustering)}

\CommentTok{\# Output}
\FunctionTok{cat}\NormalTok{(}\StringTok{"Clustering coefficient of the original network: "}\NormalTok{, original\_clustering, }\StringTok{"}\SpecialCharTok{\textbackslash{}n}\StringTok{"}\NormalTok{)}
\end{Highlighting}
\end{Shaded}

\begin{verbatim}
## Clustering coefficient of the original network:  0.429691
\end{verbatim}

\begin{Shaded}
\begin{Highlighting}[]
\FunctionTok{cat}\NormalTok{(}\StringTok{"Mean clustering coefficient of the configuration network: "}\NormalTok{, }\FunctionTok{mean}\NormalTok{(simulated\_clustering), }\StringTok{"}\SpecialCharTok{\textbackslash{}n}\StringTok{"}\NormalTok{)}
\end{Highlighting}
\end{Shaded}

\begin{verbatim}
## Mean clustering coefficient of the configuration network:  0.03444034
\end{verbatim}

\begin{Shaded}
\begin{Highlighting}[]
\FunctionTok{cat}\NormalTok{(}\StringTok{"p value:"}\NormalTok{, p\_value, }\StringTok{"}\SpecialCharTok{\textbackslash{}n}\StringTok{"}\NormalTok{)}
\end{Highlighting}
\end{Shaded}

\begin{verbatim}
## p value: 0
\end{verbatim}

We tested whether the clustering coefficient of the original network can
be explained solely by its degree distribution, by comparing it to 1000
configuration model networks with the same degree sequence. The original
network's clustering coefficient was \textbf{0.4297}, while the average
clustering coefficient from the configuration models was
\textbf{0.0342}. The p-value is \textbf{0}, indicating that none of the
simulated networks reached the original clustering level.

\textbf{Therefore, we reject the null hypothesis} and conclude that the
clustering structure in the original network \textbf{cannot be
explained} by degree distribution alone --- it has significant
non-random structure.

\subsection{8. Visualize the neighborhood of the node with the largest
centrality
(closeness)}\label{visualize-the-neighborhood-of-the-node-with-the-largest-centrality-closeness}

\begin{Shaded}
\begin{Highlighting}[]
\FunctionTok{which.max}\NormalTok{(}\FunctionTok{closeness}\NormalTok{(graph))}
\end{Highlighting}
\end{Shaded}

\begin{verbatim}
## SLAG WOOL.ROCK WOOL AND SIMILAR MINERAL WOOLS 
##                                           453
\end{verbatim}

We discovered that the node with the largest centrality/closeness is
``SLAG WOOL.ROCK WOOL AND SIMILAR MINERAL WOOLS''.

\begin{Shaded}
\begin{Highlighting}[]
\FunctionTok{neighbors}\NormalTok{(graph,}\StringTok{"SLAG WOOL.ROCK WOOL AND SIMILAR MINERAL WOOLS"}\NormalTok{)}
\end{Highlighting}
\end{Shaded}

\begin{verbatim}
## + 27/774 vertices, named, from 15427f0:
##  [1] TRAILERS & SPECIALLY DESIGNED CONTAINERS          
##  [2] PARTS OF THE MACHINERY OF 723.41 TO 723.46        
##  [3] MATERIALS OF RUBBER(E.G.,PASTES.PLATES,SHEETS,ETC)
##  [4] OTHER VEHICLES,NOT MECHANICALLY PROPELLED,PARTS   
##  [5] PARTS OF THE MACHINERY OF 744.2-                  
##  [6] MISCELLANEOUS ART.OF MATERIALS OF DIV.58          
##  [7] POULTRY, LIVE (I.E., FOWLS, DUCKS, GEESE, ETC.)   
##  [8] COLOUR.PREPTNS OF A KIND USED IN CERAMIC,ENAMELLI.
##  [9] VARNISHES AND LACOUERS;DISTEMPERS,WATER PIGMENTS  
## [10] FABRICS OF GLASS FIBRE,PILE FAB.TULLE,LACE,KNITTED
## + ... omitted several vertices
\end{verbatim}

\begin{Shaded}
\begin{Highlighting}[]
\NormalTok{neigh\_graph }\OtherTok{\textless{}{-}} \FunctionTok{make\_neighborhood\_graph}\NormalTok{(graph, }\AttributeTok{order =} \DecValTok{1}\NormalTok{, }\StringTok{"SLAG WOOL.ROCK WOOL AND SIMILAR MINERAL WOOLS"}\NormalTok{)[[}\DecValTok{1}\NormalTok{]]}

\CommentTok{\# without the following plot won\textquotesingle{}t work}
\FunctionTok{V}\NormalTok{(neigh\_graph)}\SpecialCharTok{$}\NormalTok{color }\OtherTok{\textless{}{-}} \FunctionTok{trimws}\NormalTok{(}\FunctionTok{V}\NormalTok{(neigh\_graph)}\SpecialCharTok{$}\NormalTok{color)}
\FunctionTok{E}\NormalTok{(neigh\_graph)}\SpecialCharTok{$}\NormalTok{color }\OtherTok{\textless{}{-}} \FunctionTok{trimws}\NormalTok{(}\FunctionTok{E}\NormalTok{(neigh\_graph)}\SpecialCharTok{$}\NormalTok{color)}

\FunctionTok{plot}\NormalTok{(neigh\_graph)}
\end{Highlighting}
\end{Shaded}

\includegraphics{export-network_files/figure-latex/unnamed-chunk-16-1.pdf}

\begin{Shaded}
\begin{Highlighting}[]
\NormalTok{g }\OtherTok{\textless{}{-}} \FunctionTok{as\_tbl\_graph}\NormalTok{(neigh\_graph)}

\FunctionTok{ggraph}\NormalTok{(g, }\AttributeTok{layout =} \StringTok{\textquotesingle{}fr\textquotesingle{}}\NormalTok{) }\SpecialCharTok{+}
  \FunctionTok{geom\_edge\_link}\NormalTok{(}\AttributeTok{alpha =} \FloatTok{0.5}\NormalTok{, }\AttributeTok{width =} \DecValTok{1}\NormalTok{) }\SpecialCharTok{+}
  \FunctionTok{geom\_node\_point}\NormalTok{(}\AttributeTok{size =} \DecValTok{5}\NormalTok{, }\FunctionTok{aes}\NormalTok{(}\AttributeTok{color =}\NormalTok{ name }\SpecialCharTok{==} \StringTok{"SLAG WOOL.ROCK WOOL AND SIMILAR MINERAL WOOLS"}\NormalTok{)) }\SpecialCharTok{+}
  \FunctionTok{geom\_node\_text}\NormalTok{(}\FunctionTok{aes}\NormalTok{(}\AttributeTok{label =}\NormalTok{ name), }\AttributeTok{repel =} \ConstantTok{TRUE}\NormalTok{, }\AttributeTok{size =} \DecValTok{3}\NormalTok{, }\AttributeTok{color =} \StringTok{"steelblue4"}\NormalTok{) }\SpecialCharTok{+}
  \FunctionTok{guides}\NormalTok{(}\AttributeTok{color =} \StringTok{"none"}\NormalTok{) }\SpecialCharTok{+}    
  \FunctionTok{theme\_void}\NormalTok{()}\SpecialCharTok{+}
  \FunctionTok{labs}\NormalTok{(}\AttributeTok{title =} \StringTok{"Neighborhood of Node with Highest Closeness Centrality"}\NormalTok{)}
\end{Highlighting}
\end{Shaded}

\includegraphics{export-network_files/figure-latex/unnamed-chunk-16-2.pdf}

We also used the ``visNetwork'' package to make an interactive graph:

\begin{Shaded}
\begin{Highlighting}[]
\FunctionTok{library}\NormalTok{(visNetwork)}
\NormalTok{nodes }\OtherTok{\textless{}{-}} \FunctionTok{data.frame}\NormalTok{(}\AttributeTok{id =} \FunctionTok{V}\NormalTok{(g)}\SpecialCharTok{$}\NormalTok{name, }\AttributeTok{label =} \FunctionTok{V}\NormalTok{(g)}\SpecialCharTok{$}\NormalTok{name)}
\NormalTok{edges }\OtherTok{\textless{}{-}} \FunctionTok{data.frame}\NormalTok{(}\AttributeTok{from =} \FunctionTok{as.character}\NormalTok{(}\FunctionTok{ends}\NormalTok{(g, }\FunctionTok{E}\NormalTok{(g))[,}\DecValTok{1}\NormalTok{]), }\AttributeTok{to =} \FunctionTok{as.character}\NormalTok{(}\FunctionTok{ends}\NormalTok{(g, }\FunctionTok{E}\NormalTok{(g))[,}\DecValTok{2}\NormalTok{]))}

\FunctionTok{visNetwork}\NormalTok{(nodes, edges) }\SpecialCharTok{\%\textgreater{}\%}
  \FunctionTok{visEdges}\NormalTok{(}\AttributeTok{arrows =} \StringTok{\textquotesingle{}to\textquotesingle{}}\NormalTok{, }\AttributeTok{color =} \FunctionTok{list}\NormalTok{(}\AttributeTok{color =} \StringTok{"gray"}\NormalTok{, }\AttributeTok{hover =} \StringTok{"red"}\NormalTok{)) }\SpecialCharTok{\%\textgreater{}\%}
  \FunctionTok{visNodes}\NormalTok{(}\AttributeTok{size =} \DecValTok{15}\NormalTok{, }\AttributeTok{color =} \FunctionTok{list}\NormalTok{(}\AttributeTok{background =} \StringTok{"lightblue"}\NormalTok{, }\AttributeTok{border =} \StringTok{"darkblue"}\NormalTok{)) }\SpecialCharTok{\%\textgreater{}\%}
  \FunctionTok{visOptions}\NormalTok{(}\AttributeTok{highlightNearest =} \ConstantTok{TRUE}\NormalTok{, }\AttributeTok{nodesIdSelection =} \ConstantTok{TRUE}\NormalTok{) }\SpecialCharTok{\%\textgreater{}\%}
  \FunctionTok{visLayout}\NormalTok{(}\AttributeTok{randomSeed =} \DecValTok{123}\NormalTok{)}
\end{Highlighting}
\end{Shaded}

\begin{verbatim}
## file:////private/var/folders/5k/zh67w_yx2fj10t7z34l2ct7m0000gn/T/Rtmp7TOv65/file3bdf3f1d26e4/widget3bdf12e4631b.html screenshot completed
\end{verbatim}

\includegraphics{export-network_files/figure-latex/unnamed-chunk-17-1.pdf}

\end{document}
